			\section{My Projects} 
        
\textbf{ROADEF2018} 
	\hfill February 2018 -- January 2019\\
		\textsl{Optimization contest}. Bi-annual optimization competition of the French Society of Operations Research (ROADEF). The topic was Cutting Optimization Problem. The code can be found at https://github.com/pchtsp/roadef2018.\\
		More information at: http://www.roadef.org/challenge/2018/en/index.php \\
		\textsl{Technologies: python, cython.}

\textbf{OPTIMA-Dassault} 
	\hfill May 2018 -- now\\
		Collaboration project together with Dassault Aviation to develop a method to help plan the maintenances of a fleet of military aicraft. The types of maintenances A, B, C taking into account capacity limitations. \\
		\textsl{Technologies: python, R.}

\textbf{OPTIPLAN2018} 
	\hfill March 2018 -- June 2018\\
		\textsl{Optimization contest}. Contest organized by the French Army. The topic was Maintenance Planning for Military Aircraft. I participated as a judge during the technical presentations done by the final contestants.

\textbf{MOPTA2017} 	
	\hfill March 2017 -- August 2017\\
		\textsl{Optimization contest}. Anual optimization modelling competition organized by Lehigh University, USA. The topic that year was Production and Delivery of Radio-Pharmaceuticalsto Medical Imaging Centers. The code and mathematical model is publicly available at the following github page:\\
		https://github.com/pchtsp/mopta2017

\textbf{General Motors: Real-time inventory and production simulation} 
	\hfill January 2015 -- August 2017\\
	Specification, design, contruction and deployment of a multi-user client-server application used to simulate the future production of cars, calculate consumptions, integrate arrivals of parts and give warnings about future inventory problems based on hourly data about arrivals and the status of the plant.
	\textsl{Technologies: AIMMS, AIMMS PRO, R.}

\textbf{Alstom: Train simulator} 
	\hfill February 2013 -- August 2017\\
	Specification, design, contruction and deployment of a desktop application used to construct and simulate railway systems in an graphical, easy-to-use way. The application includes the arrival and flow of passengers across the system, the movement of trains with acceleration and decceleration and complex security logic among many other functionalities. It outputs precise, minute-by-minute information about a whole day worth of services.
	\textsl{Technologies: SIMIO, R, C\#.}

\textbf{Repsol: Liquid Petrouleum Gas Logistics} 
	\hfill September 2015 -- August 2017\\
	Migration, improvement and deployment of a client-server enterprise application used to optimize all the movement, buying and production of liquid gas across Spain. Both in yearly (month by month) and monthly (day by day) scopes.
	\textsl{Technologies: AIMMS, AIMMS PRO, Access, SQL Server, ILOG.}

\textbf{Air Liquide: Production scheduling optimization} 
	\hfill March 2013 -- June 2013\\
	Design, development and deployment of a desktop application to optimize monthly production scheduling of industrial gases based on technical requirements, energy prices and regulatory legislation by choosing, hourly, the production mode.
	\textsl{Technologies: AIMMS, Excel.}

\textbf{CLH: Rostering Optimization and management} 
	\hfill June 2011 -- December 2013\\
	Specification, design and deployment of a multi-user application to help monthly workforce planning in Spain's airports for an oil logistics company, optimizing resources given a specific hourly demand, labor legislation and workers preferences.
	\textsl{Technologies: AIMMS, Excel, Oracle, SAP.}

% \textbf{Subete.pe}
% 	\hfill 2010 -- 2011\\
% 	\textsl{Carpooling initiative} 
% 	Project aimed at getting people to share car rides from and to the university by creating a social network based on the university community. This initiative, backed up and financed by the PUCP in Lima Peru, wants to ultimately achieve an instant dynamic ride-sharing experience in which users will be able to, on the go, find people to travel with.

